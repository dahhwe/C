Пятая практическая работа по дисциплине — Системное программирование.

\subsection*{Цель работы}

Изучение особенностей межпроцессного взаимодействия в ОС G\+N\+U/\+Linux.

\subsection*{Задачи}


\begin{DoxyEnumerate}
\item Ознакомиться с краткими теоретическими сведениями по организации межпроцессного взаимодействия в ОС G\+N\+U/\+Linux.
\item Получить у преподавателя собственный вариант задания, который предусматривает разработку серверной и клиентской частей приложения, взаимодействующих посредством механизма Internet-\/сокетов и сетевых протоколов. Использование высокоуровневых средств является ошибочным. При выполнении заданий с нечетным вариантом должны использоваться потоковые сокеты, с четным вариантом — дейтаграммные. Обеспечить сборку обеих частей программы, как отдельно, так и полностью, с использованием инструментального набора G\+NU Autotools.
\item Используя изученные механизмы, разработать и отладить\+:
\end{DoxyEnumerate}
\begin{DoxyItemize}
\item Серверную часть;
\item Клиентскую часть.
\end{DoxyItemize}
\begin{DoxyEnumerate}
\item Составить общее описание результатов, инструкции по сборке и использованию программы, а также инструкцию по получению документации, сформировать архив формата tar.\+gz и представить на проверку с исходными текстами программы. Внимание\+: исходные тексты программ должны соответствовать принятому стандарту кодирования, а также содержать комментарии в стиле системы Doxygen.
\end{DoxyEnumerate}

\subsection*{Вариант №24}

Клиент принимает от пользователя строку, содержащую дату в формате ДД.\+ММ.\+ГГГГ и отсылает ее серверу. Необходимо учесть ввод некорректных значений (например, 36.\+14.\+2106 или 04.\+02.\+1918). Кроме того, в 1900, 1930 и 1931 году был день 29 февраля. В качестве базовой даты принять 1 января 1601. Сервер получает строку с датой и выводит на экран день недели, соответствующий введенной дате (для 08.\+04.\+2013 — «понедельник» или «monday»). 